\documentclass{article}
\usepackage[utf8]{inputenc}
\usepackage{caption}
\usepackage[margin=1in]{geometry}
\usepackage{graphicx}

\begin{document}
\begin{titlepage}

\centering
\vspace*{2cm}
{\Huge System Design Document\par}
\vspace{1cm}
{\LARGE Course Evaluation System\par}
\vspace{1cm}
{\Large Client: Dr. Harlan Onsrud\par}
\vspace{1cm}
{\Large Team EVAL\par}
\vspace{.2cm}
{\Large Jovon Craig, Sam Elliott, Robert Judkins, and Stanley Small\par}
\vspace{1cm}
{\Large November 16, 2018\par}
\vspace{.5cm}

\end{titlepage}

\newpage

\begin{center}
{\includegraphics[scale=.2]{images/team_logo.png}} \\ 	\bigskip
{\LARGE Course Evaluation System } \\ \medskip
{\large System Design Document } \\ \medskip
\end{center}

\tableofcontents

\newpage

\section{Introduction}
\subsection{Purpose of This Document}

This system design document gives an overview of the structure of our course evaluation system. The first section describes the system architecture, how the components function, and how the components relate to each other. The second section covers the design of the data, detailing the database schema and the properties of the files that are used by the system. The document's third section has a table that lists the components that fulfill each functional requirement.

This document is intended for the development team, the product client, Dr. Harlan Onsrud, and potential users of the system. Team EVAL needs this document to ensure that the product works as intended. Dr. Onsrud needs it to know that the program that he desires will be fully realized. The document also helps the software's users in that they can learn more about the functions and architecture of the evaluation system.

\subsection{References}

Craig, J., Elliott, S., Judkins, R., \& Small, S. 29 October 2018. System Requirements Specification (Rep.).
\vspace{3mm}\newline
Craig, J., Elliott, S., Judkins, R., \& Small, S. 30 November 2018. User Interface Design Document (Rep.).
\vspace{3mm}\newline
Fowler, M. (2004). UML Distilled: A Brief Guide to the Standard Object Modeling Language. Boston:

Addison-Wesley.

\section{System Architecture}
\subsection{Architectural Design}

Figure 1 is an abstraction of the proposed system architecture. With this figure, we aim to communicate what components are in the system and the API that links the components together.

\begin{center}
\captionof{figure}{}
\label{fig:componentdiagram}
{\includegraphics[scale=.7]{images/component_diagram.png}} 
\end{center}

\vspace{3mm}

LimeSurvey (version 3.15.3+181108) is running on an Apache web server that uses PHP 5.6. LimeSurvey maintains a database that runs MySQL 5.5.61. The team will use Amazon Web Services to host the evaluation system on the Web.

The system consists of three main components, all connected with an API. The first component is the front end, which handles the user interface, and displays the visuals and back-end data. It will be written in Javascript and use the React library. The back end communicates with a MySQL database, which collects survey templates and responses. It will be written in Python, and its endpoints will use the Flask library.

A single API will connect the front end to the back end and the back end to LimeSurvey. A REST interface connects the front end to the back end using endpoints. Separate functions will utilize the LimeSurvey API so that the back end can access the LimeSurvey database, which our system's code will not touch. A user will have no contact will LimeSurvey and will only communicate with the front end. Because it will not be modified, the database maintained by LimeSurvey is not included in Figure~\ref{fig:sequencediagram}. 

\subsection{Decomposition Description}

Figure 2 abstracts the major functions that are expected to be in the system. This sequence diagram illustrates a typical session and is meant to communicate a more detailed view of the components and their relationships. The ``Database'' refers to the back-end database.

\begin{center}
\captionof{figure}{}
\label{fig:sequencediagram}
{\includegraphics[scale=.62]{images/sequence_diagram.png}} 
\end{center}

To begin using the course evaluation system, the instructor must first log in with a username and password (``log in''). The API retrieves the user's token from the database to confirm that the password is correct (``get token'', ``authenticate''). Next, the instructor creates an evaluation form to send to students (``create form''). The API adds an empty course to the database (``add form'', ``post new course''). The instructor then edits the survey with the appropriate info and saves the form (``edit form'', ``save form''). Upon saving, the API updates the database with the entered information (``update form'', ``put course'').

With an evaluation form completed, the instructor can then publish it online (``publish survey''). The API retrieves the course information from the database to translate the survey form into a tab-separated .txt file (``get course'', ``translate to txt''). It then sends the .txt file to LimeSurvey and tells the survey software to message the students through e-mail (``export form'', ``message students''). The API retrieves the students' responses, translates them from another tab-separated .txt file, and stores them into the database (``translate from txt'', ``post results''). Finally, the instructor can view the survey results, with the help of the API getting the results from the database and finding their averages (``retrieve results'', ``get results'', ``compute averages'').

\section{Persistent Data Design}
\subsection{Database Descriptions}

Our database will include tables that store information about courses, survey questions, and survey responses. A diagram of the database schema is shown below:

\begin{center}
\captionof{figure}{}
\label{fig:schemadiagram}
{\includegraphics[scale=.5]{images/schema_diagram.png}} 
\end{center}

The database contains six object tables and two relationship tables. The \textbf{Survey} table contains survey IDs, the surveys' URLs (generated by LimeSurvey) and associated instructor IDs. The e-mail addresses to send the surveys to are stored in the \textbf{E-mail} table. The \textbf{Instructor} table contains the names of the instructors along with authentication tokens generated by Google OAuth. The \textbf{Tag} table contains additional information (e.g. course names) about the surveys. These tags are used to categorize and search surveys. The \textbf{Question} table contains all questions to be entered into the surveys, along with help text and the type of question. The \textbf{Response} table includes students' responses to certain questions. The relationship tables, \textbf{Survey-to-Question} and \textbf{Survey-to-Tag}, link the surveys to their appropriate questions and tags.

\subsection{File Descriptions}

The system requires several pieces of data, including database files, front-end markup, and LimeSurvey files, to run as intended. A file structure diagram is given below:

FILE STRUCTURE DIAGRAM - STAN

Files are stored in the back-end database to keep track of all course information, survey forms, and survey responses that users have input. The schema above shows how the files' data is to be organized. To reduce the risk of a breach, any data present in the database for 60 days gets removed by the system. There are also mark-up files, which specify the look and feel of the front end. These are static and permanent, and thus do not require maintenance.

Lastly, the system briefly stores LimeSurvey files in the form of tab-separated text documents. They either contain survey form data or student responses. The form data files are deleted immediately after they are sent to students, and the response data files are deleted immediately after they are stored in the database.

\section{Requirements Matrix}

The following table lists the functions in our system, as shown in the sequence diagram, that meet each functional requirement given in the system requirements specification:

\begin{center}
\captionof{table}{}
\begin{tabular}{|p{3.2cm}|p{3.2cm}|p{6cm}|} 
\hline
\textbf{Use Case Number} & \textbf{Use Case Name} & \textbf{System Component(s)} \\
\hline
1 & Log on to system & log in, get token\\ 
\hline
2 & Create course & create form, add form, post new course\\ 
\hline
3 & Edit course info & edit form\\ 
\hline
4 & Save course info & save form, update form, put course\\ 
\hline
5 & Publish survey & publish survey, get course, translate to txt\\ 
\hline
6 & Notify students & export form, message students\\ 
\hline
7 & View survey results & view results, retrieve results, get results, compute averages\\ 
\hline
\end{tabular}
\end{center}

\appendix

\newpage
\section{Agreement Between Customer and Contractor}
This page shows that all members of Team EVAL and the client, Harlan Onsrud, have agreed on all the information in the system design document. By signing this document, Team EVAL and Dr. Onsrud agree on the system's architecture, components, relations between the components, database schema, required files, and file descriptions.

The team will follow a process in the case that the deesign document is changed after we sign it. First, the team writes a rough draft of the changes to be made to the document. Second, all team members and Harlan Onsrud will sign the document agreeing to the changes. Finally, the changes are made to the final copy of the document.

\vspace{.7in}
\noindent
\begin{tabular}{ p{5cm} p{5cm} p{5cm} } 
\textbf{\textit{Name}} & \textbf{\textit{Signature}} & \textbf{\textit{Date}} \\[.5cm]
\textbf{Jovon Craig} & $\rule{5cm}{.1mm}$ & $\rule{5cm}{.1mm}$\\[.5cm]
\textbf{Sam Elliott} & $\rule{5cm}{.1mm}$ & $\rule{5cm}{.1mm}$\\[.5cm]
\textbf{Robert Judkins} & $\rule{5cm}{.1mm}$ & $\rule{5cm}{.1mm}$\\[.5cm]
\textbf{Stanley Small} & $\rule{5cm}{.1mm}$ & $\rule{5cm}{.1mm}$\\[.5cm]
\textbf{Harlan Onsrud} & $\rule{5cm}{.1mm}$ & $\rule{5cm}{.1mm}$\\[.5cm]
Customer Comments: & \multicolumn{2}{ l }{ $\rule{10.45cm}{.1mm}$ }\\[.5cm]
\multicolumn{3}{ l }{ $\rule{15.9cm}{.1mm}$ }\\[.5cm]
\end{tabular}

\newpage
\section{Team Review Sign-off}

This page shows that all members of Team EVAL have reviewed the system design document and agreed on its content. By signing this document, the team members agree on that all information about the system's architecture and design are accurate. There is nothing in the document that is a source of contention.

\vspace{.7in}
\noindent
\begin{tabular}{ p{5cm} p{5cm} p{5cm} } 
\textbf{\textit{Name}} & \textbf{\textit{Signature}} & \textbf{\textit{Date}} \\[.5cm]
\textbf{Jovon Craig} & $\rule{5cm}{.1mm}$ & $\rule{5cm}{.1mm}$\\[.5cm]
Comments: & \multicolumn{2}{ l }{ $\rule{10.45cm}{.1mm}$ }\\[.5cm]
\multicolumn{3}{ l }{ $\rule{15.9cm}{.1mm}$ }\\[.5cm]
\textbf{Sam Elliott} & $\rule{5cm}{.1mm}$ & $\rule{5cm}{.1mm}$\\[.5cm]
Comments: & \multicolumn{2}{ l }{ $\rule{10.45cm}{.1mm}$ }\\[.5cm]
\multicolumn{3}{ l }{ $\rule{15.9cm}{.1mm}$ }\\[.5cm]
\textbf{Robert Judkins} & $\rule{5cm}{.1mm}$ & $\rule{5cm}{.1mm}$\\[.5cm]
Comments: & \multicolumn{2}{ l }{ $\rule{10.45cm}{.1mm}$ }\\[.5cm]
\multicolumn{3}{ l }{ $\rule{15.9cm}{.1mm}$ }\\[.5cm]
\textbf{Stanley Small} & $\rule{5cm}{.1mm}$ & $\rule{5cm}{.1mm}$\\[.5cm]
Comments: & \multicolumn{2}{ l }{ $\rule{10.45cm}{.1mm}$ }\\[.5cm]
\multicolumn{3}{ l }{ $\rule{15.9cm}{.1mm}$ }\\[.5cm]
\end{tabular}


\newpage
\section{Document Contributions}

Stanley Small converted the system design document template to LaTeX for our own document and included a template of the appendices. He wrote a draft of the title page, architectural design section, and database description. He also added the component diagram and helped with the database schema. Stan contributed approximately ?? percent of the document.

Jovon Craig wrote the purpose of the document, references, decomposition description, file descriptions, and requirements matrix. He made revisions to the title page, architectural design section, and database description. He also added the sequence diagram and made a draft of the database schema. Jovon contributed about 40 percent of the document.

Sam Elliott helped revise the database schema and made some minor revisions to the system design document. Sam contributed about ?? percent of the document.

Robert Judkins ??. Robert contributed about ?? percent of the document.

\end{document}

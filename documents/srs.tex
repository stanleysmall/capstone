\documentclass{article}
\usepackage[utf8]{inputenc}
\usepackage{caption}
\usepackage[margin=1in]{geometry}
\title{System Requirements Specification}
\author{Team EVAL}
\date{October ??, 2018}

\begin{document}

\maketitle

\newpage

\begin{center}
[Put team logo here]\\ \bigskip
{\LARGE College Course Evaluation System }\\ \medskip
{\large System Requirements Specification }\\
\end{center}

\tableofcontents

\newpage

\section{Introduction}
\subsection{Purpose of This Document}

This system requirements specification details what our course evaluation system does and what tests we will make to ensure the system is complete. It includes why we are creating the system, the scope of the product, diagrams that illustrate the system, what we will deliver to the customer, and currently pending issues. This document is intended for the product client, Harlan Onsrud, and potential users of the system.

\subsection{References}
\subsection{Purpose of the Product}

The University of Maine gives out course evaluation surveys to students at the end of each course. The survey is filled on a bubble sheet and is then scanned. Harlan Onsrud finds it inconvenient for the school administrators to manually scan and compile the survey results. He desires an online, automated evaluation system to improve productivity.\par

With this new product, course teachers and administrators can customize their lists of questions for course evaluations and send the lists out to students. Students will receive an e-mail telling them to complete the survey, and they will be periodically notified if the survey is incomplete. After survey submissions, the teachers and administrators can view the average of the results for each question over a certain course, instructor, department, and university.

\subsection{Product Scope}

\section{Functional Requirements}

The functional requirements specify what actions the program will perform. Each requirement is represented as a use case.

\begin{center}
\vspace{4in}
\captionof{table}{}
\begin{tabular}{|p{3.5cm}|p{7.5cm}|} 
\hline
\textbf{Number} & 1  \\
\hline
\textbf{Name} & Set up courses  \\ 
\hline
\textbf{Summary} & An administrator or teacher adds the surveyed courses along with the student rolls \\ 
\hline
\textbf{Priority} & how critical this use case is to the customer (1 to 5, 5 being most critical for delivery)\\ 
\hline
\textbf{Preconditions }& conditions that must be true before the use case trigger \\ 
\hline
\textbf{Postconditions} & conditions that will be true after the use case completes \\ 
\hline
\textbf{Primary Actor }& Administrator, Teacher \\ 
\hline
\textbf{Secondary Actors} & other systems that are relied upon to accomplish the use case \\ 
\hline
\textbf{Trigger }& the action that starts the use case \\ 
\hline
\textbf{Main Scenario }& 
\begin{tabular}{l|p{5.8cm}} 
\textbf{Step }& \textbf{Action}\\
\hline
1 & steps of the use case from trigger to goal delivery \\
\hline
2 & \\
\hline
3 & \\
\end{tabular}\\ 
\hline
\textbf{Extensions }&
\begin{tabular}{l|p{5.8cm}} 
\textbf{Step }& \textbf{Branching Action}\\
\hline
1a & condition causing branching  : action or name of sub use case  \\
\end{tabular}\\
\hline
\textbf{Open Issues} & list of issues awaiting decisions that affect the use case \\ 
\hline
\end{tabular}

\bigskip
\captionof{table}{}
\begin{tabular}{|p{3.5cm}|p{7.5cm}|} 
\hline
\textbf{Number} & 2  \\
\hline
\textbf{Name} & Edit survey questions  \\ 
\hline
\textbf{Summary} & An administrator or teacher edits the survey questions for the courses \\ 
\hline
\textbf{Priority} & how critical this use case is to the customer (1 to 5, 5 being most critical for delivery)\\ 
\hline
\textbf{Preconditions }& conditions that must be true before the use case trigger \\ 
\hline
\textbf{Postconditions} & conditions that will be true after the use case completes \\ 
\hline
\textbf{Primary Actor }& Administrator, Teacher \\ 
\hline
\textbf{Secondary Actors} & other systems that are relied upon to accomplish the use case \\ 
\hline
\textbf{Trigger }& the action that starts the use case \\ 
\hline
\textbf{Main Scenario }& 
\begin{tabular}{l|p{5.8cm}} 
\textbf{Step }& \textbf{Action}\\
\hline
1 & steps of the use case from trigger to goal delivery \\
\hline
2 & \\
\hline
3 & \\
\end{tabular}\\ 
\hline
\textbf{Extensions }&
\begin{tabular}{l|p{5.8cm}} 
\textbf{Step }& \textbf{Branching Action}\\
\hline
1a & condition causing branching  : action or name of sub use case  \\
\end{tabular}\\
\hline
\textbf{Open Issues} & list of issues awaiting decisions that affect the use case \\ 
\hline
\end{tabular}

\bigskip
\captionof{table}{}
\begin{tabular}{|p{3.5cm}|p{7.5cm}|} 
\hline
\textbf{Number} & 3  \\
\hline
\textbf{Name} & Store questions list preset  \\ 
\hline
\textbf{Summary} & An administrator or teacher stores a list of questions as a preset \\ 
\hline
\textbf{Priority} & how critical this use case is to the customer (1 to 5, 5 being most critical for delivery)\\ 
\hline
\textbf{Preconditions }& conditions that must be true before the use case trigger \\ 
\hline
\textbf{Postconditions} & conditions that will be true after the use case completes \\ 
\hline
\textbf{Primary Actor }& Administrator, Teacher \\ 
\hline
\textbf{Secondary Actors} & other systems that are relied upon to accomplish the use case \\ 
\hline
\textbf{Trigger }& the action that starts the use case \\ 
\hline
\textbf{Main Scenario }& 
\begin{tabular}{l|p{5.8cm}} 
\textbf{Step }& \textbf{Action}\\
\hline
1 & steps of the use case from trigger to goal delivery \\
\hline
2 & \\
\hline
3 & \\
\end{tabular}\\ 
\hline
\textbf{Extensions }&
\begin{tabular}{l|p{5.8cm}} 
\textbf{Step }& \textbf{Branching Action}\\
\hline
1a & condition causing branching  : action or name of sub use case  \\
\end{tabular}\\
\hline
\textbf{Open Issues} & list of issues awaiting decisions that affect the use case \\ 
\hline
\end{tabular}

\bigskip
\captionof{table}{}
\begin{tabular}{|p{3.5cm}|p{7.5cm}|} 
\hline
\textbf{Number} & 4  \\
\hline
\textbf{Name} & View survey results  \\ 
\hline
\textbf{Summary} & An administrator or teacher views averages of the survey results \\ 
\hline
\textbf{Priority} & how critical this use case is to the customer (1 to 5, 5 being most critical for delivery)\\ 
\hline
\textbf{Preconditions }& conditions that must be true before the use case trigger \\ 
\hline
\textbf{Postconditions} & conditions that will be true after the use case completes \\ 
\hline
\textbf{Primary Actor }& Administrator, Teacher \\ 
\hline
\textbf{Secondary Actors} & other systems that are relied upon to accomplish the use case \\ 
\hline
\textbf{Trigger }& the action that starts the use case \\ 
\hline
\textbf{Main Scenario }& 
\begin{tabular}{l|p{5.8cm}} 
\textbf{Step }& \textbf{Action}\\
\hline
1 & steps of the use case from trigger to goal delivery \\
\hline
2 & \\
\hline
3 & \\
\end{tabular}\\ 
\hline
\textbf{Extensions }&
\begin{tabular}{l|p{5.8cm}} 
\textbf{Step }& \textbf{Branching Action}\\
\hline
1a & condition causing branching  : action or name of sub use case  \\
\end{tabular}\\
\hline
\textbf{Open Issues} & list of issues awaiting decisions that affect the use case \\ 
\hline
\end{tabular}

\bigskip
\vspace{2.6in}
\captionof{table}{}
\begin{tabular}{|p{3.5cm}|p{7.5cm}|} 
\hline
\textbf{Number} & 5 \\
\hline
\textbf{Name} & Notify students  \\ 
\hline
\textbf{Summary} & A teacher e-mails students about the survey, giving automatic reminders to complete it \\ 
\hline
\textbf{Priority} & how critical this use case is to the customer (1 to 5, 5 being most critical for delivery)\\ 
\hline
\textbf{Preconditions }& conditions that must be true before the use case trigger \\ 
\hline
\textbf{Postconditions} & conditions that will be true after the use case completes \\ 
\hline
\textbf{Primary Actor }& Teacher \\ 
\hline
\textbf{Secondary Actors} & other systems that are relied upon to accomplish the use case \\ 
\hline
\textbf{Trigger }& the action that starts the use case \\ 
\hline
\textbf{Main Scenario }& 
\begin{tabular}{l|p{5.8cm}} 
\textbf{Step }& \textbf{Action}\\
\hline
1 & steps of the use case from trigger to goal delivery \\
\hline
2 & \\
\hline
3 & \\
\end{tabular}\\ 
\hline
\textbf{Extensions }&
\begin{tabular}{l|p{5.8cm}} 
\textbf{Step }& \textbf{Branching Action}\\
\hline
1a & condition causing branching  : action or name of sub use case  \\
\end{tabular}\\
\hline
\textbf{Open Issues} & list of issues awaiting decisions that affect the use case \\ 
\hline
\end{tabular}
\end{center}

\subsection{Tests}

These are the tests that will verify the functional requirements:

\begin{enumerate}
  \item Test
  \item Test
  \item Test
  \item Test
  \item Test
\end{enumerate}

\section{Non-Functional Requirements}

The non-functional requirements state the qualities of the program that are unrelated to its function.

\begin{center}
\captionof{table}{}
\begin{tabular}{|p{3.5cm}|p{7.5cm}|} 
\hline
\textbf{Number} & 1  \\
\hline
\textbf{Priority} & 3\\ 
\hline
\textbf{Description} & The software should be supported by Windows, Mac, Linux, iOS, and Android. \\ 
\hline
\textbf{Tests }& the test number(s) corresponding to the requirement \\ 
\hline
\end{tabular}

\bigskip
\captionof{table}{}
\begin{tabular}{|p{3.5cm}|p{7.5cm}|} 
\hline
\textbf{Number} & 2  \\
\hline
\textbf{Priority} & 4\\ 
\hline
\textbf{Description} & The software should be accessible through Safari, Chrome, Firefox, and Edge. \\ 
\hline
\textbf{Tests }& the test number(s) corresponding to the requirement \\ 
\hline
\end{tabular}


\bigskip
\captionof{table}{}
\begin{tabular}{|p{3.5cm}|p{7.5cm}|} 
\hline
\textbf{Number} & 3  \\
\hline
\textbf{Priority} & 5\\ 
\hline
\textbf{Description} & All questions entered by the teacher or administrator shall appear on the output survey. \\ 
\hline
\textbf{Tests }& the test number(s) corresponding to the requirement \\ 
\hline
\end{tabular}


\bigskip
\captionof{table}{}
\begin{tabular}{|p{3.5cm}|p{7.5cm}|} 
\hline
\textbf{Number} & 4  \\
\hline
\textbf{Priority} & 5\\ 
\hline
\textbf{Description} & All data stored in the program's database shall be valid. \\ 
\hline
\textbf{Tests }& the test number(s) corresponding to the requirement \\ 
\hline
\end{tabular}


\bigskip
\captionof{table}{}
\begin{tabular}{|p{3.5cm}|p{7.5cm}|} 
\hline
\textbf{Number} & 5  \\
\hline
\textbf{Priority} & 5\\ 
\hline
\textbf{Description} & All collected survey data shall not be alterable.\\ 
\hline
\textbf{Tests }& the test number(s) corresponding to the requirement \\ 
\hline
\end{tabular}


\bigskip
\captionof{table}{}
\begin{tabular}{|p{3.5cm}|p{7.5cm}|} 
\hline
\textbf{Number} & 6  \\
\hline
\textbf{Priority} & 4\\ 
\hline
\textbf{Description} & Teachers shall not be able to access data of courses other than their own. \\ 
\hline
\textbf{Tests }& the test number(s) corresponding to the requirement \\ 
\hline
\end{tabular}


\bigskip
\captionof{table}{}
\begin{tabular}{|p{3.5cm}|p{7.5cm}|} 
\hline
\textbf{Number} & 7  \\
\hline
\textbf{Priority} & 3\\ 
\hline
\textbf{Description} & The mean time between failures should be at least 60 minutes. \\ 
\hline
\textbf{Tests }& the test number(s) corresponding to the requirement \\ 
\hline
\end{tabular}


\bigskip
\captionof{table}{}
\begin{tabular}{|p{3.5cm}|p{7.5cm}|} 
\hline
\textbf{Number} & 8  \\
\hline
\textbf{Priority} & 5\\ 
\hline
\textbf{Description} & Students shall have no access to any data stored by the program. \\ 
\hline
\textbf{Tests }& the test number(s) corresponding to the requirement \\ 
\hline
\end{tabular}


\bigskip
\captionof{table}{}
\begin{tabular}{|p{3.5cm}|p{7.5cm}|} 
\hline
\textbf{Number} & 9  \\
\hline
\textbf{Priority} & 5 \\ 
\hline
\textbf{Description} & All survey responses shall be anonymous. \\ 
\hline
\textbf{Tests }& the test number(s) corresponding to the requirement \\ 
\hline
\end{tabular}


\bigskip
\captionof{table}{}
\begin{tabular}{|p{3.5cm}|p{7.5cm}|} 
\hline
\textbf{Number} & 10  \\
\hline
\textbf{Priority} & 2\\ 
\hline
\textbf{Description} & The software should scale to at least three universites, 1000 courses per semester, 1000 teachers per university, and 500 students per course. \\ 
\hline
\textbf{Tests }& the test number(s) corresponding to the requirement \\ 
\hline
\end{tabular}


\bigskip
\captionof{table}{}
\begin{tabular}{|p{3.5cm}|p{7.5cm}|} 
\hline
\textbf{Number} & 11  \\
\hline
\textbf{Priority} & 1 \\ 
\hline
\textbf{Description} & The software should not exceed 500 MB in size. \\ 
\hline
\textbf{Tests }& the test number(s) corresponding to the requirement \\ 
\hline
\end{tabular}


\bigskip
\captionof{table}{}
\begin{tabular}{|p{3.5cm}|p{7.5cm}|} 
\hline
\textbf{Number} & 12  \\
\hline
\textbf{Priority} & 4\\ 
\hline
\textbf{Description} & The software's source code shall be open-source. \\ 
\hline
\textbf{Tests }& the test number(s) corresponding to the requirement \\ 
\hline
\end{tabular}


\bigskip
\captionof{table}{}
\begin{tabular}{|p{3.5cm}|p{7.5cm}|} 
\hline
\textbf{Number} & 13  \\
\hline
\textbf{Priority} & 4 \\ 
\hline
\textbf{Description} & The licensing requirements of any non-original code shall be met.\\ 
\hline
\textbf{Tests }& the test number(s) corresponding to the requirement \\ 
\hline
\end{tabular}

\bigskip
\vspace{.5in}
\captionof{table}{}
\begin{tabular}{|p{3.5cm}|p{7.5cm}|} 
\hline
\textbf{Number} & 14  \\
\hline
\textbf{Priority} & 4 \\ 
\hline
\textbf{Description} & The software shall meet UMaine AFUM requirements.\\ 
\hline
\textbf{Tests }& the test number(s) corresponding to the requirement \\ 
\hline
\end{tabular}

\end{center}

\subsection{Tests}

These are the tests that will verify the non-functional requirements:

\begin{enumerate}
  \item Test
  \item Test
  \item Test
  \item Test
  \item Test
\end{enumerate}

\section{User Interface}

See ``User Interface Design Document for the College Course Evaluation System.''

\section{Deliverables}

The following lists the date and format that each submission will be delivered:

\begin{center}
\captionof{table}{}
\begin{tabular}{|p{6cm}|p{3cm}|p{3cm}|} 
\hline
\textbf{Submission} & \textbf{Date of Delivery} & \textbf{Format} \\
\hline
System Requirements Specification & ?/??/201? & Hard Copy\\ 
\hline
System Design Document & ?/??/201? & Hard Copy\\ 
\hline
User Interface Design Document & ?/??/201? & Hard Copy\\ 
\hline
User Manual & ?/??/201? & Hard Copy\\ 
\hline
Administration Manual & ?/??/201? & Hard Copy\\ 
\hline
System Requirements Specification & ?/??/201? & Electronic\\ 
\hline
System Design Document & ?/??/201? & Electronic\\ 
\hline
User Interface Design Document & ?/??/201? & Electronic\\ 
\hline
User Manual & ?/??/201? & Electronic\\ 
\hline
Administration Manual & ?/??/201? & Electronic\\ 
\hline
Source Code & ?/??/201? & Electronic\\ 
\hline
Web Link to Program & ?/??/201? & Electronic\\ 
\hline
\end{tabular}
\end{center}

\section{Open Issues}

The most significant issue our team has relates to the non-anonymous student responses. In a course evaluation survey, a student has the option to add a signed comment to be stored in an instructor's file. However, the LimeSurvey software does not store the identities of survey respondents. We need to find a way to collect the signed comment along with the survey, while ensuring that the signature is authentic.

\appendix

\newpage
\section{Agreement Between Customer and Contractor}
\newpage
\section{Team Review Sign-off}
\newpage
\section{Document Contributions}
\end{document}

